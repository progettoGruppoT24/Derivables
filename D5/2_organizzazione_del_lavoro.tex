\section{Organizzazione del lavoro}
Fin da subito si sono decise due cose:
\begin{itemize}
    \item le ore sarebbero state segnate su un foglio di calcolo condiviso (si trova nella repository dei derivable);
    \item la scrittura dei derivable sarebbe avvenuta utilizzando overleaf e quindi \LaTeX.
\end{itemize}
Abbiamo cercato di vederci e di discutere di persona del progetto il martedì, mercoledì e il giovedì. Gli altri giorni continuavamo comunque a comunicare online via whatsapp. In alcuni casi (non pochi) ci siamo sentiti per parlare a voce su discord. \\
Per quanto riguarda la divisione dei compiti. Per il primo derivable non ci siamo divisi i ruoli in maniera specializzata e ogni membro ha fatto un po' di tutto rispettando le tempistiche delle scadenze interne. A partire dal D2 in poi, la complessità dei derivable ci ha imposto una divisione più rigida dei compiti, abbiamo diviso tra i vari membri del gruppo determinate parti del derivable da "scrivere" e da correggere. Questa divisione ci ha permesso di finire i vari derivable sempre in tempo eccetto qualche giorno di ritardo con il D2. Abbiamo inoltre aumentato le sessioni di confronto tra tutti i membri per garantire che ognuno fosse completamente al corrente dello stato del progetto e che si fosse tutti d'accordo sulle scelte prese in merito a quest'ultimo. \\

%Andando avanti con il progetto in particolare dalla metà del D2 in poi, abbiamo iniziato ad organizzarci con più precisione. Nessuno faceva più o meno tutto, ma abbiamo iniziato a dividerci i compiti. Una, massimo due componenti del gruppo si dedicavano ad uno specifico argomento e poi gli altri si occupavano di correggere eventuali errori rileggendo quello che era stato scritto, progettato o implementato e ponendo una serie di domande a chi si era occupato di quella specifica cosa. Si è cercato in questo modo di generare una discussione e di risolvere eventuali dubbi.
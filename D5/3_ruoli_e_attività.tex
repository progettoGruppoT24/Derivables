\section{Ruoli e attività}
In questa sezione sono mostrati i ruoli che ogni componente ha svolto all'interno del gruppo, con le principali attività di cui ognuno si è occupato di svolgere per ogni derivable.
\\
\\
\begin{tabularx}{\linewidth}{
    |>{\hsize=.8\hsize}X|% 10% of 3\hsize 
    >{\hsize=0.8\hsize}X|% 30% of 3\hsize
    >{\hsize=1.4\hsize}X|% 30% of 3\hsize 
       % sum=3.0\hsize for 3 columns
  }
    \hline
    \textbf{Componente del team} & \textbf{Ruolo} & \textbf{Principali attività} \\
    \hline
    Lorenzo D'Ambrosio & Sviluppatore & Ha contribuito alla realizzazione dei primi due documenti e ha avuto un ruolo fondamentale nello sviluppo dell'applicazione, sia lato back-end che front-end sia nella realizzazione delle pagine che nella cura della loro estetica (css). Questo membro ha preso parte alla realizzazione del D1, D2, e D4 in maniera più sostanziosa. \\
    \hline
    Andrea Goldoni & Progettista / Sviluppatore & Ha contribuito alla scrittura in \LaTeX \,di una buona parte dei documenti, ha preso parte alla maggior parte dei diagrammi, dagli use-case del D2 ai diagrammi del D4 infine ha contribuito sia allo sviluppo di alcune API che alle funzioni di gioco e alle relative pagine html/javascript del front-end. Infine si è occupato della documentazione su swagger e della scrittura dei test delle API con Jest. Questo membro del gruppo ha preso parte integralmente in tutti i derivable. \\
    \hline
    Marco Antonio Murru & Project leader / Progettista / Analista & Questo membro del gruppo si è occupato di definire e analizzare le funzionalità del progetto ad alto livello, realizzare i diagrammi dei vari documenti, ha contribuito alla stesura in \LaTeX \,dei documenti, e infine si è occupato di verificare il corretto funzionamento dell'applicazione una volta implementata. Questo membro del gruppo ha preso parte integralmente in tutti i derivable. \\
    \hline
\end{tabularx}

%GOLDO: D1: si è occupato di scrivere qualche requisito funzionale, non funzionale, di disegnare qualche mockup dell'applicazione, di descrivere la parte di front-end e di back-end. \newline D2: si è occupato di, descrivere una parte dei RF con use-case diagram, descrivere una parte dei RNF con tabelle strutturate, descrive soltanto le entità del diagramma di contesto, di progettare metà del diagramma delle componenti e di descrivere una piccola parte (1/3) delle componenti. \newline D3: si è occupato di, progettare la parte del diagramma delle classi per il quale aveva progettato il diagramma delle componenti e di scriverne i relativi OCL. \newline D4: si è occupato di disegnare l'user flow, sviluppare la parte di front-end e back-end che riguarda la classifica e il single-player, scrivere la documentazione delle API con swagger e di effettuare il testing delle API. \newline D5: scrittura che riguarda la propria parte.
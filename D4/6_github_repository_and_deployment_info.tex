\section{GitHub Repository And Deployment Info}

Per strutturare il progetto abbiamo deciso di utilizzare 3 repositories all'interno di Github: Derivables (contenente i vari derivables da consegnare), Front-End (contenente tutti i file di Front-End, compresi file .css, .html e .javascript, quest'ultimo contenente gli script utilzizati dalle pagine html) e Back-End (contenente tutti i file che permettono l'utilizzo, l'implementazione e la documentazione delle API, il testing e i collegamenti al database MongoDB).
\\
\\
Per quanto riguarda il deployment, abbiamo scelto di utilizzare \href{https://railway.app/}{railway}, che offre un servizio di deployment gratuito, per un totale di 500 ore (attivo fino al 21/01/2023).
\\
Di seguito sono elencati i link per raggiungere le cartelle di Github ed il deployment dell'applicazione:
\begin{itemize}
    \item \href{https://github.com/progettoGruppoT24}{Link Github gruppo}
    \item \href{https://github.com/progettoGruppoT24/Back-End}{Link repository Back-End}
    \item \href{https://github.com/progettoGruppoT24/Front-End}{Link repository Front-End}
    \item \href{https://github.com/progettoGruppoT24/Derivables}{Link repository Derivables}
    \item \href{https://jguesser-backend-unitn.up.railway.app/api-docs/}{Link swagger}
    \item \href{https://jguesser-backend-unitn.up.railway.app/}{Link deployment Back-End}
    \item \href{https://jguesser-unitn-se-group24.up.railway.app/}{Link deployment Front-End}
\end{itemize}